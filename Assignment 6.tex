\documentclass[journal,12pt,twocolumn]{IEEEtran}

\usepackage{setspace}
\usepackage{gensymb}
\singlespacing
\usepackage[cmex10]{amsmath}

\usepackage{amsthm}

\usepackage{multirow}
\usepackage{tikz}
\usetikzlibrary{matrix}
\tikzset{ 
table/.style={
  matrix of nodes,
  row sep=-\pgflinewidth,
  column sep=-\pgflinewidth,
  nodes={rectangle,text width=1em,align=center},
  text depth=1.25ex,
  text height=2.5ex,
  nodes in empty cells
},
row 1/.style={nodes={fill=green!10,text depth=0.4ex,text height=2ex}},
row 6/.style={nodes={text depth=0.4ex,text height=2ex}},
column 2/.style={nodes={text width=30ex,text height=2ex}},
column 3/.style={nodes={text width=18ex,text height=2ex}},
column 4/.style={nodes={text width=10ex,text height=2ex}},
column 1/.style={nodes={fill=green!10}},
}
\counterwithin{equation}{section}
\singlespacing

\usepackage{mathrsfs}
\usepackage{txfonts}
\usepackage{stfloats}
\usepackage{bm}
\usepackage{cite}
\usepackage{cases}
\usepackage{subfig}

\usepackage{longtable}
\usepackage{multirow}

\usepackage{enumitem}
\usepackage{mathtools}
\usepackage{steinmetz}
\usepackage{tikz}
\usepackage{circuitikz}
\usepackage{verbatim}
\usepackage{tfrupee}
\usepackage[breaklinks=true]{hyperref}
\usepackage{graphicx}
\usepackage{tkz-euclide}

\usetikzlibrary{calc,math}
\usepackage{listings}
    \usepackage{color}                                            %%
    \usepackage{array}                                            %%
    \usepackage{longtable}                                        %%
    \usepackage{calc}                                             %%
    \usepackage{multirow}                                         %%
    \usepackage{hhline}                                           %%
    \usepackage{ifthen}                                           %%
    \usepackage{lscape}     
\usepackage{multicol}
\usepackage{chngcntr}

\DeclareMathOperator*{\Res}{Res}

\renewcommand\thesection{\arabic{section}}
\renewcommand\thesubsection{\thesection.\arabic{subsection}}
\renewcommand\thesubsubsection{\thesubsection.\arabic{subsubsection}}

\renewcommand\thesectiondis{\arabic{section}}
\renewcommand\thesubsectiondis{\thesectiondis.\arabic{subsection}}
\renewcommand\thesubsubsectiondis{\thesubsectiondis.\arabic{subsubsection}}


\hyphenation{op-tical net-works semi-conduc-tor}
\def\inputGnumericTable{}                                 %%

\lstset{
%language=C,
frame=single, 
breaklines=true,
columns=fullflexible
}
\begin{document}


\newtheorem{theorem}{Theorem}[section]
\newtheorem{problem}{Problem}
\newtheorem{proposition}{Proposition}[section]
\newtheorem{lemma}{Lemma}[section]
\newtheorem{corollary}[theorem]{Corollary}
\newtheorem{example}{Example}[section]
\newtheorem{definition}[problem]{Definition}

\newcommand\Myperm[2][^n]{\prescript{#1\mkern-2.5mu}{}P_{#2}}
\newcommand\Mycomb[2][^n]{\prescript{#1\mkern-0.5mu}{}C_{#2}}
\newcommand{\BEQA}{\begin{eqnarray}}
\newcommand{\EEQA}{\end{eqnarray}}
\newcommand{\define}{\stackrel{\triangle}{=}}
\bibliographystyle{IEEEtran}
\raggedbottom
\setlength{\parindent}{0pt}
\providecommand{\mbf}{\mathbf}
\providecommand{\pr}[1]{\ensuremath{\Pr\left(#1\right)}}
\providecommand{\qfunc}[1]{\ensuremath{Q\left(#1\right)}}
\providecommand{\sbrak}[1]{\ensuremath{{}\left[#1\right]}}
\providecommand{\lsbrak}[1]{\ensuremath{{}\left[#1\right.}}
\providecommand{\rsbrak}[1]{\ensuremath{{}\left.#1\right]}}
\providecommand{\brak}[1]{\ensuremath{\left(#1\right)}}
\providecommand{\lbrak}[1]{\ensuremath{\left(#1\right.}}
\providecommand{\rbrak}[1]{\ensuremath{\left.#1\right)}}
\providecommand{\cbrak}[1]{\ensuremath{\left\{#1\right\}}}
\providecommand{\lcbrak}[1]{\ensuremath{\left\{#1\right.}}
\providecommand{\rcbrak}[1]{\ensuremath{\left.#1\right\}}}
\theoremstyle{remark}
\newtheorem{rem}{Remark}
\newcommand{\sgn}{\mathop{\mathrm{sgn}}}
\providecommand{\abs}[1]{\left\vert#1\right\vert}
\providecommand{\res}[1]{\Res\displaylimits_{#1}} 
\providecommand{\norm}[1]{\left\lVert#1\right\rVert}
%\providecommand{\norm}[1]{\lVert#1\rVert}
\providecommand{\mtx}[1]{\mathbf{#1}}
\providecommand{\mean}[1]{E\left[ #1 \right]}
\providecommand{\fourier}{\overset{\mathcal{F}}{ \rightleftharpoons}}
%\providecommand{\hilbert}{\overset{\mathcal{H}}{ \rightleftharpoons}}
\providecommand{\system}{\overset{\mathcal{H}}{ \longleftrightarrow}}
	%\newcommand{\solution}[2]{\textbf{Solution:}{#1}}
\newcommand{\solution}{\noindent \textbf{Solution: }}
\newcommand{\cosec}{\,\text{cosec}\,}
\providecommand{\dec}[2]{\ensuremath{\overset{#1}{\underset{#2}{\gtrless}}}}
\newcommand{\myvec}[1]{\ensuremath{\begin{pmatrix}#1\end{pmatrix}}}
\newcommand{\mydet}[1]{\ensuremath{\begin{vmatrix}#1\end{vmatrix}}}
\numberwithin{equation}{subsection}
\makeatletter
\@addtoreset{figure}{problem}
\makeatother
\let\StandardTheFigure\thefigure
\let\vec\mathbf
\renewcommand{\thefigure}{\theproblem}
\def\putbox#1#2#3{\makebox[0in][l]{\makebox[#1][l]{}\raisebox{\baselineskip}[0in][0in]{\raisebox{#2}[0in][0in]{#3}}}}
     \def\rightbox#1{\makebox[0in][r]{#1}}
     \def\centbox#1{\makebox[0in]{#1}}
     \def\topbox#1{\raisebox{-\baselineskip}[0in][0in]{#1}}
     \def\midbox#1{\raisebox{-0.5\baselineskip}[0in][0in]{#1}}
\vspace{3cm}
\title{AI5002: Assignment 5}
\author{Pradyumn Sharma\\ AI21MTECH02001}
\maketitle
\newpage
\bigskip
\renewcommand{\thefigure}{\theenumi}
\renewcommand{\thetable}{\theenumi}
%
latex codes from 
%
\begin{lstlisting}
https://github.com/96143/Assignment-5/blob/main/Assignment%205.tex
\end{lstlisting}

\section{Problem}
Let X denote the sum of the numbers obtained
when two fair dice are rolled. Find the
variance and standard deviation of X
\section{Solution}
When two fair dice are rolled, 6×6=36 observations are obtained.\\
Let X denote the sum of the numbers obtained when two fair dice are rolled.\\
So, X may have values 2, 3, 4, 5, 6, 7, 8, 9, 10, 11 and 12
\\
\begin{center}
 \begin{table}[ht]
\begin{tabular}{|l|l|l|l|}
\hline
\textbf{X}    &\textbf{Outcomes}    & \textbf{No of Outcomes}   & \textbf{Probability}  \\ \hline
2 & (1,1) & 1 & 1/36 \\ \hline
3 & (1,2),(2,1) & 2 & 2/36 \\ \hline
4 & (1,3),(2,2),(3,1) & 3 & 3/36 \\ \hline
5 & (1,4),(2,3),(3,2),(4,1) & 4 & 4/36\\\hline
6 & (1,5),(2,4),(3,3),(4,2),(5,1) & 5 & 5/36 \\\hline
7 & (1,6),(2,5),(3,4),(4,3),(5,2),(6,1) & 6 & 6/36 \\\hline
8 & (2,6),(3,5),(4,4),(5,3),(6,2) & 5 & 5/36 \\\hline
9 & (3,6),(4,5),(5,4),(6,3) & 4 & 4/36 \\\hline
10 & (4,6),(5,5),(6,4) & 3 & 3/36 \\\hline
11 & (5,6),(6,5) & 2 & 2/36 \\\hline
12 & (6,6) & 1 & 1/36 \\  \hline
\end{tabular}
\end{table}
\end{center}
Thus,\\
The probability distribution table is \\
\begin{center}
 \begin{table}[ht]
\begin{tabular}{|l|l|l|l|l|l|l|l|l|l|l|l|}
\hline
\textbf{X}    &2    & 3   & 4   & 5    & 6   & 7    & 8   &9    & 10    & 11    &12       \\ \hline
\textbf{P(X)} & 1/36 & 1/18 & 1/12 & 1/9  & 5/36 & 1/6 & 5/36 & 1/9 & 1/12 & 1/18 & 1/36 \\ \hline
\end{tabular}
\end{table}
\end{center}
Now we have to find the variance.
\begin{equation}\label{eq:2.0.1}
     Var(X) = E[X^2] - (E[X])^2
\end{equation}
Finding E[X]
\begin{equation}\label{eq:2.0.2}
    E[X] = \sum_{i=1}^{n} x_i p_i
\end{equation}
\begin{equation*}
\begin{split}
    E[X] = 2\times \frac{1}{36} + 3 \times \frac{1}{18} + 4\times \frac{1}{12} + 5 \times \frac{1}{9}\\ + 6 \times \frac{5}{36} + 7 \times \frac{1}{6} + 8 \times \frac{5}{36} + 9 \times \frac{1}{9} \\+ 10 \times \frac{1}{12} + 11 \times \frac{1}{18} + 12 \times \frac{1}{36}
\end{split}
\end{equation*}
\begin{equation*}
  \implies  E[X] = 7
\end{equation*}
\newpage
Finding $E[X^2]$
\begin{equation}\label{eq:2.0.3}
    E[X^2] = \sum_{i=1}^{n} (x_i)^2 p_i
\end{equation}
\begin{equation*}
\begin{split}
    E[X^2] = 2^2\times \frac{1}{36} + 3^2 \times \frac{1}{18} + 4^2\times \frac{1}{12} + 5^2 \times \frac{1}{9}\\ + 6^2 \times \frac{5}{36} + 7^2 \times \frac{1}{6} + 8^2 \times \frac{5}{36} + 9^2 \times \frac{1}{9} \\+ 10^2 \times \frac{1}{12} + 11^2 \times \frac{1}{18} + 12^2 \times \frac{1}{36}
\end{split}
\end{equation*}
\begin{equation*}
  \implies  E[X^2] = \frac{329}{6}
\end{equation*}
Using \eqref{eq:2.0.1} Variance is given by \\
\begin{equation*}
    Var(X) = \frac{329}{6} - (7)^2
\end{equation*}
\begin{equation*}
    Var(X) = \frac{35}{6} = 5.83
\end{equation*}
Standard Deviation is given by
\begin{equation}\label{eq:2.0.4}
    \sigma_x = \sqrt{Var(X)} 
\end{equation}
\begin{equation*}
    \sigma_x  = \sqrt{5.83} = 2.415
\end{equation*}
\end{document}

