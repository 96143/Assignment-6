\documentclass[journal,12pt,twocolumn]{IEEEtran}

\usepackage{setspace}
\usepackage{gensymb}
\singlespacing
\usepackage[cmex10]{amsmath}

\usepackage{amsthm}

\usepackage{mathrsfs}
\usepackage{txfonts}
\usepackage{stfloats}
\usepackage{bm}
\usepackage{cite}
\usepackage{cases}
\usepackage{subfig}

\usepackage{longtable}
\usepackage{multirow}

\usepackage{enumitem}
\usepackage{mathtools}
\usepackage{steinmetz}
\usepackage{tikz}
\usepackage{circuitikz}
\usepackage{verbatim}
\usepackage{tfrupee}
\usepackage[breaklinks=true]{hyperref}
\usepackage{graphicx}
\usepackage{tkz-euclide}

\usetikzlibrary{calc,math}
\usepackage{listings}
    \usepackage{color}                                            %%
    \usepackage{array}                                            %%
    \usepackage{longtable}                                        %%
    \usepackage{calc}                                             %%
    \usepackage{multirow}                                         %%
    \usepackage{hhline}                                           %%
    \usepackage{ifthen}                                           %%
    \usepackage{lscape}     
\usepackage{multicol}
\usepackage{chngcntr}

\DeclareMathOperator*{\Res}{Res}

\renewcommand\thesection{\arabic{section}}
\renewcommand\thesubsection{\thesection.\arabic{subsection}}
\renewcommand\thesubsubsection{\thesubsection.\arabic{subsubsection}}

\renewcommand\thesectiondis{\arabic{section}}
\renewcommand\thesubsectiondis{\thesectiondis.\arabic{subsection}}
\renewcommand\thesubsubsectiondis{\thesubsectiondis.\arabic{subsubsection}}


\hyphenation{op-tical net-works semi-conduc-tor}
\def\inputGnumericTable{}                                 %%

\lstset{
%language=C,
frame=single, 
breaklines=true,
columns=fullflexible
}
\begin{document}


\newtheorem{theorem}{Theorem}[section]
\newtheorem{problem}{Problem}
\newtheorem{proposition}{Proposition}[section]
\newtheorem{lemma}{Lemma}[section]
\newtheorem{corollary}[theorem]{Corollary}
\newtheorem{example}{Example}[section]
\newtheorem{definition}[problem]{Definition}

\newcommand\Myperm[2][^n]{\prescript{#1\mkern-2.5mu}{}P_{#2}}
\newcommand\Mycomb[2][^n]{\prescript{#1\mkern-0.5mu}{}C_{#2}}
\newcommand{\BEQA}{\begin{eqnarray}}
\newcommand{\EEQA}{\end{eqnarray}}
\newcommand{\define}{\stackrel{\triangle}{=}}
\bibliographystyle{IEEEtran}
\raggedbottom
\setlength{\parindent}{0pt}
\providecommand{\mbf}{\mathbf}
\providecommand{\pr}[1]{\ensuremath{\Pr\left(#1\right)}}
\providecommand{\qfunc}[1]{\ensuremath{Q\left(#1\right)}}
\providecommand{\sbrak}[1]{\ensuremath{{}\left[#1\right]}}
\providecommand{\lsbrak}[1]{\ensuremath{{}\left[#1\right.}}
\providecommand{\rsbrak}[1]{\ensuremath{{}\left.#1\right]}}
\providecommand{\brak}[1]{\ensuremath{\left(#1\right)}}
\providecommand{\lbrak}[1]{\ensuremath{\left(#1\right.}}
\providecommand{\rbrak}[1]{\ensuremath{\left.#1\right)}}
\providecommand{\cbrak}[1]{\ensuremath{\left\{#1\right\}}}
\providecommand{\lcbrak}[1]{\ensuremath{\left\{#1\right.}}
\providecommand{\rcbrak}[1]{\ensuremath{\left.#1\right\}}}
\theoremstyle{remark}
\newtheorem{rem}{Remark}
\newcommand{\sgn}{\mathop{\mathrm{sgn}}}
\providecommand{\abs}[1]{\left\vert#1\right\vert}
\providecommand{\res}[1]{\Res\displaylimits_{#1}} 
\providecommand{\norm}[1]{\left\lVert#1\right\rVert}
%\providecommand{\norm}[1]{\lVert#1\rVert}
\providecommand{\mtx}[1]{\mathbf{#1}}
\providecommand{\mean}[1]{E\left[ #1 \right]}
\providecommand{\fourier}{\overset{\mathcal{F}}{ \rightleftharpoons}}
%\providecommand{\hilbert}{\overset{\mathcal{H}}{ \rightleftharpoons}}
\providecommand{\system}{\overset{\mathcal{H}}{ \longleftrightarrow}}
	%\newcommand{\solution}[2]{\textbf{Solution:}{#1}}
\newcommand{\solution}{\noindent \textbf{Solution: }}
\newcommand{\cosec}{\,\text{cosec}\,}
\providecommand{\dec}[2]{\ensuremath{\overset{#1}{\underset{#2}{\gtrless}}}}
\newcommand{\myvec}[1]{\ensuremath{\begin{pmatrix}#1\end{pmatrix}}}
\newcommand{\mydet}[1]{\ensuremath{\begin{vmatrix}#1\end{vmatrix}}}
\numberwithin{equation}{subsection}
\makeatletter
\@addtoreset{figure}{problem}
\makeatother
\let\StandardTheFigure\thefigure
\let\vec\mathbf
\renewcommand{\thefigure}{\theproblem}
\def\putbox#1#2#3{\makebox[0in][l]{\makebox[#1][l]{}\raisebox{\baselineskip}[0in][0in]{\raisebox{#2}[0in][0in]{#3}}}}
     \def\rightbox#1{\makebox[0in][r]{#1}}
     \def\centbox#1{\makebox[0in]{#1}}
     \def\topbox#1{\raisebox{-\baselineskip}[0in][0in]{#1}}
     \def\midbox#1{\raisebox{-0.5\baselineskip}[0in][0in]{#1}}
\vspace{3cm}
\title{AI5002: Assignment 6}
\author{Pradyumn Sharma\\ AI21MTECH02001}
\maketitle
\newpage
\bigskip
\renewcommand{\thefigure}{\theenumi}
\renewcommand{\thetable}{\theenumi}
%
latex codes from 
%
\begin{lstlisting}
https://github.com/96143/Assignment-3/tree/main
\end{lstlisting}
\section{Problem}
An urn contains 25 balls of which 10 balls
bear a mark ’X’ and the remaining 15 bear a
mark ’Y’. A ball is drawn at random from the
urn, its mark is noted down and it is replaced.
If 6 balls are drawn in this way, find the
probability that
\begin{enumerate}
    \item  all will bear ’X’ mark.
    \item  not more than 2 will bear ’Y’ mark.
    \item  at least one ball will bear ’Y’ mark.
    \item  the number of balls with ’X’ mark and
’Y’ mark will be equal.
\end{enumerate}
\section{Solution}
Let X be the number of balls with mark 'X' \\
Drawing a ball is a Bernoulli trial\\ 
So X has a Binomial distribution
\begin{equation}\label{eq:1}
    P(X=x) = \Mycomb[n]{x} p^x q^{n-x}
\end{equation}
Here, \\
number of balls drawn = n = 6\\
probability of getting ball with 'X' mark = p = $\frac{10}{25}$ = $\frac{2}{5}$\\
q = 1 - p = $\frac{3}{5}$\\
Hence, 
\begin{equation}\label{eq:2}
    P(X=x) = \Mycomb[6]{x} \left(\frac{2}{5}\right)^x \left(\frac{3}{5}\right)^{6-x}
\end{equation}
\begin{enumerate}
    \item probability that all will bear ’X’ mark
    probability that all will bear ’X’ mark = P(X=6)
    Putting x = 6 in \eqref{eq:2}
\begin{alignat*}{3}
    P(X=6) = \Mycomb[6]{6} \left(\frac{2}{5}\right)^6 \left(\frac{3}{5}\right)^{6-6}\\
    = \Mycomb[6]{6} \left(\frac{2}{5}\right)^6 \left(\frac{3}{5}\right)^{0}\\
    = \left(\frac{2}{5}\right)^6
\end{alignat*}
\item probability that not more than 2 will bear ’Y’ mark
\begin{equation*}
\begin{split}
    P(not~more~than~2~’Y’)\\
    = P(6X,0Y)+P(5X,1Y)+P(4X,2Y)
\end{split}
\end{equation*}
\begin{equation*}
\begin{split}
    = \Mycomb[6]{6} \left(\frac{2}{5}\right)^6 \left(\frac{3}{5}\right)^{6-6} + \Mycomb[6]{5} \left(\frac{2}{5}\right)^5 \left(\frac{3}{5}\right)^{6-5} + \Mycomb[6]{4} \left(\frac{2}{5}\right)^4 \left(\frac{3}{5}\right)^{6-4}\\
    = \Mycomb[6]{6} \left(\frac{2}{5}\right)^6 \left(\frac{3}{5}\right)^{0} + \Mycomb[6]{5} \left(\frac{2}{5}\right)^5 \left(\frac{3}{5}\right)^{1} + \Mycomb[6]{4} \left(\frac{2}{5}\right)^4 \left(\frac{3}{5}\right)^{2}
\end{split}
\end{equation*}
\begin{equation*}
     = 7 \left(\frac{2}{5}\right)^4
\end{equation*}
\item probability that at least one ball will bear ’Y’ mark
\begin{equation*}
     P(at~least~one~ball~bear~’Y’) = 1 - P(no~ball~bear~'Y')\\
\end{equation*}
\begin{equation*}
\begin{split}
     =  1 - P(all~balls~bear~'X')\\
     = 1 - P(X = 6)
\end{split}
\end{equation*}
\begin{equation*}
\begin{split}
    = 1 - \Mycomb[6]{6} \left(\frac{2}{5}\right)^6 \left(\frac{3}{5}\right)^{6-6}\\
    = 1 - \Mycomb[6]{6} \left(\frac{2}{5}\right)^6 \left(\frac{3}{5}\right)^{0}\\
    = 1 - \left(\frac{2}{5}\right)^6
\end{split}
\end{equation*}
\item probability that the number of balls with ’X’ mark and
’Y’ mark will be equal
\begin{equation*}
\begin{split}
     P(X~\&~Y~balls~are~equal) = P(X=3)\\
    = \Mycomb[6]{3} \left(\frac{2}{5}\right)^3 \left(\frac{3}{5}\right)^{6-3}\\
    = \Mycomb[6]{6} \left(\frac{2}{5}\right)^3 \left(\frac{3}{5}\right)^{3}\\
    = 20 \left(\frac{2}{5}\right)^3 \left(\frac{3}{5}\right)^{3}\\
    = \frac{864}{3125}
\end{split}
\end{equation*}
\end{enumerate}
\end{document}
